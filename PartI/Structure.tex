% !TEX encoding = UTF-8 Unicode
% !TEX root = ../diss.tex


% -----------------------------------------------------
\chapter{Structure}

The template was designed using 

\begin{itemize}
	\item LuaHBTex, Version 1.18.0 (TeX Live 2024)
\end{itemize}

% -----------------------------------------------------
\paragraph{Configuration}

\begin{description}
	\item[Template] The main configs are stored under 
	\begin{itemize}
	 	\item \lstinline{/0_config}
	 \end{itemize} 
	 You should adapt \lstinline{/0_config/config.tex}. There is space to add all your ``custom'' packages there.
	 %
	 Currently, that holds acronyms and custom macros. I did not put the packages here, because I think keeping them as close to each other as possible makes it easier to avoid conflicts.
\end{description}





% -----------------------------------------------------
\paragraph{Bibliography}

\begin{description}
	\item [Bibliography files] Bibliography files go into 
	\begin{itemize}
		\item \lstinline{/bibliography}
	\end{itemize}
		There are two special files for the \lstinline{Publication} sections:

	\begin{itemize}
		\item \lstinline{/bibliography/own.bib} to hold bibliography of you own publications.
		\item \lstinline{/bibliography/own_openaccess.bib} to hold open-access (for instance, ePrint or arXiv) bibliography of you own publications.
	\end{itemize}
	%
	\item[Document] The bibliography is added to the document via 
	\begin{itemize}
		\item \lstinline{/FrontBackmatter/Biblipgraphy.tex}
		\item \lstinline{/FrontBackmatter/Publications.tex}
	\end{itemize}
	%
	\item [Biblatex] If you use biblated with cryptobib in a folder called (say, in a folder called \lstinline{cryptobiblink}), I strongly recommend to use 
	\href{https://github.com/thomwiggers/extract_from_bibliography/blob/main/extract_from_bibliography.py}{Extract From Bibliography}
	to extract the relevant entries via 

	\begin{lstlisting}[breaklines=true]
	python3 extract_from_bibliography.py diss.bcf cryptobiblink/crypto.bib cryptobiblink/dummy.bib  > bibliography/reduced_crypto.bib
	\end{lstlisting}
	%
	I did not figure out what the 3rd input (``dummy.bib'') does, but the above worked flawlessly for me. 
	%
	This will significantly speed-up the compilation.
\end{description}

