% !TEX encoding = UTF-8 Unicode
% !TEX root = ../diss.tex



% -----------------------------------------------------
\chapter{How to use this Template}


\lipsum[1-1]%
\footnote{\detokenize{\footnote}: This is a numbered and referenced footnote in the margin.} 
\lipsum[2-2]%
\sidenote{\detokenize{\sidenote}: This is a numbered but \textbf{un}referenced footnote in the margin.}
\sidecomment{\detokenize{\sidecomment}: This is a unnumbered and unreferenced comment in the margin.}


\paragraph{Citation and Floats}

Citations \cite{cryptoeprint:2021/1484} appear as preview in the margin. 
Floats cannot have footnotes (without modifications) and citations with \detokenize{\cite} act like footnotes in this template. 
Consequently, 
you need to use \detokenize{\citeonly} instead, 
which \emph{only} print the citation in the text but not in the margin. 
Additionally, 
if there is too little space in the margin, 
LaTeX might fail to find a valid positioning. 
This might, for example, cause \emph{latexmk} too loop infinitely (or throw an error). 
To easiest to avoid this is to limit the number of note in the margin, especially if there is a \emph{full-width figure} on the page.


Floats include (but are not limited to):
\begin{itemize}
	\item captions
	\item footnote, sidenote, sidecomment
\end{itemize}

The class offers a \emph{full} citation:
\begin{itemize}
 	\item \detokenize{\fullfullcite}
 \end{itemize} 

\fullfullcite{cryptoeprint:2020/367}

There is also a command to present your contributions:

\begin{itemize}
	\item \detokenize{\contentsource}[TEXT]{OFFICIAL CITETATION}{OPEN CITETATION}{Contribution}[IMPLEMENTATION]
\end{itemize}

\blockmargin
\begin{figure}[th]
\begin{fullwidthfig}
\contentsource[TEXT.]{DBLP:conf/ccs/TiepeltD20}{cryptoeprint:2020/367}{Everyone did everything.}[\url{Link to Github}]
\end{fullwidthfig}
\end{figure}
\unblockmargin

\FloatBarrier
\paragraph{Figures and Captions}

\begin{figure}
	\includegraphics[width=\linewidth]{example-image-a}
	%
	\caption{Linewidth Figure}
	\label{LABEL1}
\end{figure}

You might want to put \detokenize{\blockmargin}, \detokenize{\unblockmargin} around your ``fullwidth'' floats to avoid LaTeX trying to put a float in the (non-existing) margin next to it. 

\blockmargin
\begin{figure}
	\begin{fullwidthfig}
	\includegraphics[width=\linewidth]{example-image-a}
	%
	\caption{Fullwidth Figure}
	\label{LABEL2}
	\end{fullwidthfig}
\end{figure}
\unblockmargin


\begin{figure}
	\begin{sidecaption}{Sidecaption}
		%
		\includegraphics[width=\linewidth]{example-image-a}
		%
		\label{LABEL3}
	\end{sidecaption}
\end{figure}

\begin{figure}
	%
	\subbottom[{Subfigure in Memoir class}\label{LABEL4}]{
          \includegraphics[width=.7\linewidth]{example-image-a}
    }\\%
    \subbottom[{Subfigure in Memoir class}\label{LABEL5}]{
          \includegraphics[width=.7\linewidth]{example-image-b}
    }%
    \caption{Caption of subfigure in Memoir.}
    \label{LABEL6}
\end{figure}


\FloatBarrier
% -----------------------------------------------------
\paragraph{Acronyms}

This is a acronym: \gls{AES}, where only the first occurence has a hyperlink to the reference. 
\begin{itemize}
	\item This should not have a hyperref: \gls{AES}. 
	\item You can reset the reference using \detokenize{\glsresetall}, for example, after every new part. 
\glsresetall{} 
Not it should be referenced and expanded again: \gls{AES} 
	\item If you want hyperlinks on every occurrence, search for ``ACRONYMS'' in \lstinline{0_dissconfig/dissertationpackage.sty} and remove the corresponding lines. 
\end{itemize}







