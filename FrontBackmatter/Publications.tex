% !TEX encoding = UTF-8 Unicode
% !TEX root = ../diss.tex


%*******************************************************
% Publications
%*******************************************************
%\pdfbookmark[1]{Publications}{publications}
%
\chapter*{Publications}\addcontentsline{toc}{chapter}{Publications}%
%This might come in handy for PhD theses: some ideas and figures have appeared previously in the following publications:
%\noindent Put your publications from the thesis here. The packages \texttt{multibib} or \texttt{bibtopic} etc. can be used to handle multiple different bibliographies in your document.
% See section 3.8.4
% http://ftp.fau.de/ctan/macros/latex/contrib/biblatex/doc/biblatex.pdf
\begin{refsection}[bibliography/own] %The optional argument is a comma-separated list of resources specific to the reference section.
    \small
    \nocite{*} % is local to to the enclosing refsection
    {\let\clearpage\relax
    \begin{fullwidth}%
    \printbibliography[heading=none]
    \end{fullwidth}
    }
\end{refsection}

\chapter*{Open-Access Versions}

\begin{refsection}[bibliography/own_openaccess] %The optional argument is a comma-separated list of resources specific to the reference section.
    \small
    \nocite{*} % is local to to the enclosing refsection
    {\let\clearpage\relax% 
    \begin{fullwidth}%

    \printbibliography[heading=none]
    \end{fullwidth}
    }
\end{refsection}
%\emph{Attention}: This requires a separate run of \texttt{bibtex} for your \texttt{refsection}, \eg, \texttt{ClassicThesis1-blx} for this file. You might also use \texttt{biber} as the backend for \texttt{biblatex}. See also \url{http://tex.stackexchange.com/questions/128196/problem-with-refsection}.

